%% ------------- Portuguese version ------------
% \documentclass{sbrt}
% \usepackage[english,brazil]{babel}
% \usepackage[utf8]{inputenc}
% \newtheorem{theorem}{Teorema}
%% ---------------------------------------------

%% If writing in English, remove the lines above
%% and uncomment the lines below

%% ------------- English version ---------------
\documentclass[english]{sbrt}
\usepackage[english]{babel}
\usepackage[utf8]{inputenc}
\usepackage{cite}
\usepackage{graphicx}
\usepackage{float}
\usepackage[top=2cm,left=2cm,right=3cm,bottom=3cm]{geometry}
\usepackage{titlesec}
\newtheorem{theorem}{Theorem}
%% ---------------------------------------------

\titlespacing*{\section}
  {0pt}{18pt}{18pt}
  
\titlespacing*{\subsection}
  {0pt}{18pt}{18pt}

\begin{document}

\title{Multiplayer game (a decidir título)}

\author{Tiago da Silva Guerreiro and Lucas Costa dos Prazeres
    \thanks{Special thanks to professor Eduardo Cerqueira, ITEC, UFPA, Belém-PA, e-mail: cerqueira@ufpa.br}%
}

\maketitle

\baselineskip = 18pt


\begin{abstract}
    abstract.
\end{abstract}
\begin{keywords}
    keywords.
\end{keywords}

\section{\textbf{Introduction}}
Falar sobre o nosso projeto, nossos objetivos com ele e seus componentes

\section{\textbf{Background}}
Discorrer brevemente sobre websockets (como a interação ocorre, suas etapas, etc). É interessante ser coisas que citaremos na parte da simulação.

\section{\textbf{Game}}
Explicar o jogo em si e colocar algumas imagens, não deve tomar muito tempo.

\section{\textbf{Simulation}}
Fazer a simulação da perda de pacotes que o professor sugeriu, ou alguma outra interessante (talvez achemos alguma na leitura dos papers).

\section{\textbf{Conclusion}}
Concluir o trabalho daquele jeito tradicional, comentando sobre os resultados.

\cite{maodv01} Pra não dar erro.

\bibliography{sbrt}
\bibliographystyle{IEEEtran}

\end{document}
